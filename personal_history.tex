\documentclass[12pt]{article}
\usepackage[T1]{fontenc}
\usepackage[utf8]{inputenc}
\usepackage[margin=1in]{geometry}

\newcommand{\HRule}{\rule{\linewidth}{0.5mm}}
\newcommand{\Hrule}{\rule{\linewidth}{0.3mm}}

\makeatletter% since there's an at-sign (@) in the command name
\renewcommand{\@maketitle}{%
  \parindent=0pt% don't indent paragraphs in the title block
  \centering
  {\Large \bfseries\textsc{\@title}}

  \vspace{0.2in}

  {\large \textsc{\@author}}

  \vspace{0.2in}

  {\large \textsc{\@date}}
  \HRule\par%
  \par
}
\makeatother% resets the meaning of the at-sign (@)

\title{Personal History}
\author{Hilfi Alkaff}
\date{Computer Science Ph.D. Applicant}

\begin{document}
  \maketitle% prints the title block

My path to graduate school has been somewhat different than that of many students. Through my journey I have learned to not only value education, but I have also come to understand exactly what it is that I wish to do with my life. I want to contribute to the academic research community, undergraduate education and to the general society, especially the developing countries, as a professor. \newline

Growing up in one of the developing country, Indonesia, there are many societal problems that persist. For instance, the educational system at all levels is highly disorganized. During my primary and middle school years, I remembered ostensibly how memorization of formulas is sufficient to excel in the scientific subjects such as Mathematics and Physics. This exasperates me to no end as in the aforementioned subjects, an understanding of \textit{why} does the formula work should be the quintessence. I came to understand that even in higher-level education, this problem still persists. By this point, I knew that I would like to improve the quality of education in Indonesia. \newline

Since both of my parents are university lecturers in the Electrical Engineering and Computer Science (EECS) department, I have been exposed to computers and programming early in my life. Slowly, I understand the potential of computers in helping to eradicate several of the problems that exist in Indonesia. For instance, even until now, networking infrastructures have not been established properly throughout Indonesia, especially in the rural areas. This makes it challenging for me to stay connected with my relatives since they are spread throughout the country. Medical assistance could not be easily delivered to them due to the same reason. Hence, majoring in computer science will equip me with the necessary knowledge to ameliorate my home country and developing countries in general. \newline

United States has been one of the leading countries both in the field of education and computer science. Thus, at spring 2008, I enrolled at community college with the goal of transferring to University of California campuses. I felt that UC Berkeley was my ideal choice, and I worked for the next several years towards admission there. Since the cost of education is much more expensive here, I began tutoring as a means to support myself, but this job of necessity quickly turned into one of my most fulfilling experiences. \newline

I transferred to UC Berkeley in fall 2009 to finish my undergraduate education. I became immediately acquainted with the world of academic research. I realized how much dedicated researchers backed by an outstanding institution can accomplish. After surveying several research projects, I believe that the area of computer systems is the most applicable for my purpose. I decided to undertake an operating system research with the Parallel Computing Lab (ParLab) under Professor John Kubiatowicz during fall 2010 and have been involved with the project ever since. \newline

In Berkeley, I continued to be involved with education when I joined the HKN EECS honor society in spring 2010. Along with the other club members, I held tutoring hours for the undergraduates for three semesters. In fall 2010, I served as a teaching assistant for upper division operating systems class. Given the opportunity to lead discussions in class have further refined both my teaching skills and my understanding of the subject. \newline

My experiences and my home country have come to serve as a motivating factor in my life. They drive me to better myself, and to help others as much as I can. Pursuing a graduate degree gives me not only the opportunity to conduct research that I am very passionate about, but also help students learn and succeed and the society in general. I have found my path in life, and I am pursuing it. \newline

\end{document}
