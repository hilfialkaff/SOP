\documentclass[12pt]{article}
\usepackage[T1]{fontenc}
\usepackage[utf8]{inputenc}
\usepackage[margin=1in]{geometry}

\newcommand{\HRule}{\rule{\linewidth}{0.5mm}}
\newcommand{\Hrule}{\rule{\linewidth}{0.3mm}}

\makeatletter% since there's an at-sign (@) in the command name
\renewcommand{\@maketitle}{%
  \parindent=0pt% don't indent paragraphs in the title block
  \centering
  {\Large \bfseries\textsc{\@title}}

  \vspace{0.2in}

  {\large \textsc{\@author}}

  \vspace{0.2in}

  {\large \textsc{\@date}}
  \HRule\par%
  \par
}
\makeatother% resets the meaning of the at-sign (@)

\title{Statement of Purpose}
\author{Hilfi Alkaff}
\date{Computer Science Ph.D. Applicant}

\begin{document}
  \maketitle% prints the title block

%***********************************************************
%In Progress:
%***********************************************************
%
%My long-term goal is to remain in academia as a professor, contributing to both the systems research community and undergraduate education while employing my skills to further ameliorate the quality of life of humankind as a whole.
%
%From my time working as an undergraduate researcher here at UC Berkeley, I have come to understand the goals and philosophies of many of the EECS faculty. I share the vision of many of the systems professors to create real, working systems that are not only innovative, but that can also serve as a platform for further research.
%
%Coming from a developing country, I hope to be able to exercise my knowledge in order to ameliorate the quality of life in the developing countries. Many things that could be improved such as better network infrastructures to better interconnect the people. Since I'm interested in the area of operating systems and networking specifically, for me to be able to make a significant contribution, I need to possess independent research ability and be aware of the latest advancements in the systems areas. Thus, I believe that the PhD program at <SCHOOL> is the best match for my interest since the systems and networking research group has many relevant projects and world-renowned researchers.
%
% ************************************************************

For the past one and a half years, I've been conducting research under the supervision of Professor John Kubiatowicz and Dr Juan Colmenares that revolves around Berkeley's very own ManyCore OS; Tessellation. In conjunction with a graduate class project, my work in Tessellation began with designing and implementing its Quality of Service (QoS) guarantees in Linux windowing system in a team of three. We needed to implement our own lock-free shared-memory buffer for efficient communications between processes. The framework used for composing the GUI applications that we were targeting is Qt and thus, we modified its internals to encompass the same model as Tessellation. In the end, we were able to earn a 7x speedup in our version of Qt compared to the unmodified Qt when the CPU is under load. The drastic increase in performance that we achieved is enthralling since we managed to exercise our engineering skills in a commercial program used by thousands of people. Moreover, the interest expressed by companies such as Nokia when our work is presented at the ParLab 2010 winter retreat~\cite{tess_retreat2010} has further confirmed our success. \newline

Following that, I was working to complete the existing work in porting Advanced Configuration and Power Interface (ACPI) to Tessellation, collaborating with a post-doctoral researcher from NEC. Understanding the Intel manuals is a challenge in itself since since there are hardly any tutorials online and I was away from my advisors for an internship that I was undertaking on the same time. However, by the end of the summer, I'm able to navigate my way through the manuals and even managed to get ACPI running smoothly in Tessellation. When I was porting ACPI, I also restructured how PCI devices are being discovered and added support for PCI-express devices in Tessellation. As our paper deadline is approaching, my focus shifted to implementing and automating the process of running the experiments followed by parsing the results and constructing the graph. My contribution to the project has resulted in a poster presented at HotChips 2011~\cite{tess_hotchip} and two publications submitted to EuroSys and CATA 2012~\cite{tess_eurosys, tess_cata} which are currently in-review. I've accumulated abundant research experiences from Tessellation; from the early design and implementation stage up to the experimentation and writing stage. \newline

The skills that I have acquired from my research proves to be useful, even in the area of computer science that I have not explored before. In the graduate user interface class that I'm currently taking, the concepts that are taught in the class contrast with what I have learnt in my systems classes; I was required to focus on the design more rather than the implementation details. Additionally, we are also required to propose a novel project in the area computer and human interface. Having been positioned in a similar predicament in my research a few times, I quickly adapted. I started browsing through the user interface conferences such as CHI and UIST and inquired the relevant faculties to learn more about the researches that had been carried out. We discovered that user interface system that revolves around the concept of breathing has not been scrutinized thoroughly in the research community. Not only that, in the end of the class, we managed to compile a submission to the ACM CHI 2012 conference~\cite{tube_chi} which is currently under review. \newline

% -------------------------------------------------
% Optional:
% -------------------------------------------------

Interning at Qualcomm presented a distinctive challenge in planning and coordination within a large team setting. The codebase that I was working on is stupendous and intricate since my project was centered around the development of the latest android camera device drivers. In the beginning, my colleagues and I were struggling with the same bug without being aware of each other's efforts. As time progresses, we were able to coordinate much better and ensure that none of our efforts were duplicated. As the deadline of my project was looming, development accelerated considerably due to our team play. In the end of my internship, I contributed substantially to the implementation of several complex features in the device driver such as 3D snapshot \& auto-flicker detection, fixed a lot of bugs in the system and assisted Qualcomm's customers on the details of our code. Having the fruits of my project being employed by prominent companies such as Samsung confirmed its relevance to real-world problems which I hope to continue at <SCHOOL>. \newline

    Additionally, I believe that teaching is an integral part of conducting research. A discovery is far from complete if one does not also find a way to present and elucidate it to others. Even more importantly, organizing a vast array of result for presentation is a crucial skill that a researcher must share with a teacher, since without it, the researcher cannot gain a clear sense of direction of his work. \newline

    My early experience with students came as a tutor in multiple subjects such as mathematics and computer science when I was in high school and community college. Here at UC Berkeley, I've served in the HKN EECS honor society committee to hold tutoring hours for EECS undergraduates. However, my most significant and enjoyable teaching experience was as a teaching assistant for an upper-division operating system class taught by Professor John Kubiatowicz. I lead the discussion sessions and graded the projects and exams. Yet, the interesting aspect of it is during which I was holding design reviews for the projects since there are big project components to the course. It's fascinating to observe how different students perceive the projects from different angles. Having to lead them to solve the projects without telling them exactly how to do it is a challenging, but intellectually rewarding task. \newline
% -------------------------------------------------

During my time as an undergraduate, I've exposed myself to materials pertaining to computer systems. Being given the opportunity to lead paper discussions in my research meetins and the graduate system classes that I've taken have also ameliorated my critical thinking skills on pinpointing the open problems that currently exist in the area of computer systems. \newline

I look forward to continuing my research career as a PhD student. In the long-term, I wish to remain in academia as a professor that contributes to the development of education and research community. I believe that the <SCHOOL>'s PhD program is the best match for me to realize my ambition. Given my background, I believe that I am in a good position to make crucial contribution in such pursuits. \newline

\bibliographystyle{abbrv}
\bibliography{sop}

\end{document}
